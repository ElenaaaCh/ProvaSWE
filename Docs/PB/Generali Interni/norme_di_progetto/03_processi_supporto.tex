\section{Processi di Supporto}
I processi di supporto contribuiscono a rendere i processi primari più
efficienti ed efficaci.

\subsection{Documentazione}
\subsubsection{Descrizione}
Questa sezione contiene tutte le norme che ogni membro del gruppo dovrà seguire
durante la stesura della documentazione. \\Essa fornisce le indicazioni utili per
ottenere uniformità nei documenti e permette di operare seguendo linee guida
durante tutte le fasi del ciclo di vita di un documento: creazione, stesura,
verifica\textsubscript{g} ed eventuali modifiche, fino ad arrivare
all'approvazione\textsubscript{g} e quindi alla pubblicazione del documento.

\subsubsection{Ciclo di vita di un documento}
\begin{itemize}
      \item \textbf{Creazione}: viene creato un nuovo branch\textsubscript{g}, che prende il nome del documento stesso
            (vedi \nameref{inf:branch}), in cui viene caricato il template. Dal momento della creazione, in questo
            branch\textsubscript{g} ci saranno solamente le versioni verificate del documento;
      \item \textbf{Stesura}: per la stesura viene creato un branch\textsubscript{g} di lavoro, derivato dal branch del documento, nel quale si aggiungono o modificano le
            sezioni necessarie, tracciando i cambiamenti nel registro delle modifiche e aggiornando la versione;
      \item \textbf{Verifica\textsubscript{g}}: prima di considerarsi completata, ogni modifica al documento deve essere verificata da un verificatore in carica.
            Tramite una pull request\textsubscript{g}, viene messo in esame quanto aggiunto al documento rispetto alla versione precedente. Si presentano due casi:
            \begin{itemize}
                  \item \textbf{Pull request\textsubscript{g} accettata}: il verificatore non trova errori e considera il lavoro "verificato";
                        accetta quindi la pull request\textsubscript{g} e fa il merge dal branch\textsubscript{g} di lavoro al branch\textsubscript{g} del documento, chiudendo la issue assegnata e il branch di lavoro;
                  \item \textbf{Pull request\textsubscript{g} rifiutata}: il verificatore non considera la stesura adeguata e segnala errori
                        ed eventuali correzioni. La pull request\textsubscript{g} resta aperta e si continua a lavorare nel sotto-branch\textsubscript{g};
            \end{itemize}
      \item \textbf{Approvazione\textsubscript{g}}: Una volta che tutte le sezioni sono state verificate e il documento è pronto per essere pubblicato,
            si procede con l'approvazione\textsubscript{g}, effettuata dal responsabile. La procedura è la seguente:
            \begin{itemize}
                  \item Il verificatore che ha verificato l'ultima sezione completata apre una pull
                        request\textsubscript{g} per fare il merge dal branch\textsubscript{g} del
                        documento al main;
                  \item Il responsabile rilegge ed analizza il documento nella sua interezza;
                  \item Se riscontra errori o problematiche, segnala le modifiche necessarie e si
                        procede come descritto sopra;
                  \item Successivamente, aggiorna il registro delle modifiche aggiungendo una riga con
                        la versione di pubblicazione, e contrassegna il documento come "Approvato";
                  \item Nel caso di un verbale esterno, il responsabile genera il file ".pdf" e lo
                        invia al soggetto esterno con cui si è svolto l'incontro. Quest'ultimo
                        restituisce il documento firmato, che verrà caricato nella repository;
                  \item Infine, il responsabile accetta la pull request e fa il merge dal
                        branch\textsubscript{g} del documento al main.
                  \item Un'action di GitHub\textsubscript{g} si occupa di compilare il documento e di
                        renderlo disponibile pubblicamente in formato ".pdf".
            \end{itemize}
\end{itemize}

\newpage
\subsubsection{Struttura}
Ogni documento è caratterizzato da un template, che presenta queste
caratteristiche:
\begin{itemize}
      \item \textbf{Intestazione}: la prima pagina di ogni documento, contiene:
            \begin{itemize}
                  \item Logo del gruppo;
                  \item Indirizzo email del gruppo;
                  \item Titolo del documento;
                  \item Informazioni sul documento, che comprendono:
                        \begin{itemize}
                              \item Versione;
                              \item Stato: in redazione oppure approvato;
                              \item Uso: interno o esterno;
                              \item Approvazione\textsubscript{g}: indica il nome del membro del gruppo che ha
                                    approvato il documento;
                              \item Redazione: indica il nome o i nomi di chi si è occupato della stesura;
                              \item Verifica\textsubscript{g}: indica il nome del verificatore o dei verificatori;
                              \item Distribuzione: elenco delle persone o organizzazioni a cui è destinato il
                                    documento.
                        \end{itemize}
                  \item Descrizione: una breve descrizione di cosa contiene il documento;
            \end{itemize}
      \item \textbf{Registro delle modifiche}: contiene una tabella con i cambiamenti e le versioni
            attraversate dal documento prima di giungere alla versione finale. Include le seguenti colonne:
            \begin{itemize}
                  \item Versione;
                  \item Data;
                  \item Autore;
                  \item Descrizione;
                  \item Verificatore.
            \end{itemize}
      \item \textbf{Indice}: elenco ordinato dei titoli dei capitoli, per facilitare la navigazione;
      \item \textbf{Contenuto}: varia a seconda del documento.
\end{itemize}

\subsubsection{Convenzioni}
Al fine di ottenere una stesura della documentazione omogenea, e quindi più
professionale, vengono adottate le seguenti convenzioni.
\paragraph{Date}
~\\\\Per garantire un ordinamento in ordine cronologico in fase di pubblicazione,
dal più recente al meno recente, viene utilizzato il formato
\textbf{yyyy-mm-dd} se la data deve essere indicata nel nome del documento
(vale per verbali interni/esterni); per indicare una data all'interno del
documento, invece, viene utilizzato il formato \textbf{dd-mm-yyyy}.
\paragraph{Nomi di persona}
~\\\\All'interno dei documenti i nomi di persona saranno rappresentati da cognome e
nome.
\paragraph{Elenchi puntati}
~\\\\Gli elenchi puntati saranno gestiti in questo modo:
\begin{itemize}
      \item Ogni elemento dell'elenco deve iniziare con la lettera maiuscola;
      \item Ogni elemento dell'elenco deve terminare con ";", ad eccezione dell'ultimo che
            terminerà con ".";
\end{itemize}
\paragraph{Stile del testo}
\begin{itemize}
      \item \textbf{Grassetto}: utilizzato per i titoli delle sezioni, delle sottosezioni e dei paragrafi,
            e per termini specifici che necessitano di evidenza nel contesto;
      \item \textbf{Corsivo}: utilizzato per nomi di documenti, nome del proponente,
            nome del gruppo o indirizzo email del gruppo.
\end{itemize}
\paragraph{Link}
~\\\\All'interno dei documenti i link saranno strutturati in questo modo:
\begin{itemize}
      \item Breve descrizione della destinazione del link: \textbackslash
            url\{indirizzo\_del\_link\}.
\end{itemize}
\paragraph{Terminologia inglese}
~\\\\Tutti i termini in lingua inglese saranno riportati al singolare, in quanto
questa scelta risulta più coerente con l'uso comune e l'assonanza nella lingua
italiana. Ad esempio, termini come "file" o "commit" verranno sempre utilizzati
nella loro forma singolare, anche quando si riferiscono a concetti plurali, per
evitare ambiguità o traduzioni non naturali.\\ Per quanto riguarda il genere
(maschile o femminile), verrà specificato all'interno del \textit{Glossario},
così da uniformare l'interpretazione e garantire chiarezza, specialmente nei
casi in cui il termine inglese non abbia un corrispettivo diretto o il genere
non sia immediatamente evidente. Questa scelta mira a favorire una lettura
fluida e una comprensione condivisa del testo.
\paragraph{Acronimi}
~\\\\Tutti gli acronimi seguiranno queste regole:
\begin{itemize}
      \item L'acronimo deve essere scritto in maiuscolo;
      \item L'articolo davanti a un acronimo deve concordare in genere e numero con il
            termine principale che l'acronimo rappresenta. Nel caso di acronimi derivati da
            termini stranieri, la concordanza va basata sul significato tradotto o
            percepito in italiano. Questo garantisce una corretta integrazione
            dell'acronimo nella struttura grammaticale della lingua;
      \item Ogni acronimo deve essere inserito nel \textit{Glossario}, accompagnato da una
            spiegazione dettagliata del suo significato, per garantire chiarezza e
            comprensione.
\end{itemize}
\subsubsection{Strumenti per la stesura}
Per la stesura dei documenti viene usato il linguaggio LaTeX, un linguaggio di
marcatura per la preparazione di testi. È fortemente consigliato usare Visual
Studio Code, un IDE che supporta numerose estensioni, tra cui LaTeX Workshop,
per la compilazione di sorgenti ".tex" e l'anteprima del documento in formato
".pdf". \\I documenti seguono una struttura comune:
\begin{itemize}
      \item Cartella "config" contenente il file "changelog\_input" che permette di
            compilare i campi della tabella di registrazione delle modifiche;
      \item Cartella "template" contenente:
            \begin{itemize}
                  \item File "changelog": questo file contiene la definizione di un comando chiamato
                        \texttt{\char`\\changelogTable}, che serve per generare la tabella formattata;
                  \item File "package": configura pacchetti e comandi per personalizzare
                        l'impaginazione, le tabelle, le intestazioni, la numerazione e la formattazione
                        di testo, inclusi glossari e codici;
                  \item Cartella "images" contenente le immagini inserite nel documento.
            \end{itemize}
      \item File "main" include i file e i pacchetti necessari a comporre il file;
      \item File "titlepage" contenente il template della pagina di intestazione.
\end{itemize}
Oltre a questi file, viene creato un file per ogni sezione, per garantire maggiore ordine all'organizzazione
del documento e facilitare la suddivisione dei compiti.
\subsubsection{Documentazione interna}
La documentazione interna è costituita da tutti i documenti che contengono
informazioni utili per il gruppo. Tali documenti saranno comunque resi pubblici
all'interno del repository, e nella pagina web creata tramite GitHub
Pages\textsubscript{g}.\\ La documentazione interna è composta da:
\begin{itemize}
      \item \textit{\textbf{Verbali interni}}: riportano ciò che viene detto e discusso durante le riunioni interne, ossia tra i soli membri del gruppo.
            Il nome del file deve avere la forma "VI\_yyyy-mm-dd", per garantire l'ordinamento.
            \\Nei \textit{Verbali interni} la struttura del contenuto assume questa forma:
            \begin{itemize}
                  \item \textbf{Informazioni generali}: contiene i dettagli dell'incontro, nello specifico:
                        \begin{itemize}
                              \item Luogo;
                              \item Data;
                              \item Ora di inizio;
                              \item Ora di fine;
                              \item Partecipanti.
                        \end{itemize}
                  \item \textbf{Motivo della riunione}: breve descrizione narrativa di cosa è stato trattato in quell'incontro, che descrive i motivi per cui è stata indetta la riunione;
                  \item \textbf{Resoconto}: descrive i temi trattati nel dettaglio, partendo dal motivo per il quale sono stati sollevati e arrivando alla decisione presa dal gruppo a seguito di una discussione;
                  \item \textbf{Prossimi obiettivi}: elenco puntato che descrive gli obiettivi che il gruppo si impegna a portare a termine nel breve periodo, indicativamente prima della riunione successiva;
                  \item \textbf{Tracciamento delle decisioni}: tabella che riassume le decisioni prese in quella riunione indicandone:
                        \begin{itemize}
                              \item \textbf{Codice}: in formato VI Y.Z, dove Y indica il numero del \textit{Verbale} (incrementale rispetto agli altri), Z indica il numero dell'argomento trattato, per ordine di discussione;
                              \item \textbf{Descrizione}: breve descrizione dell'argomento trattato.
                        \end{itemize}
            \end{itemize}
      \item \textit{\textbf{Studio di fattibilità}}: documento interno di valutazione dei capitolati proposti dalle aziende per il progetto didattico, con lo scopo di selezionare il progetto migliore a cui candidarsi secondo valutazioni
            prese da parte del gruppo.\\
            Il documento, per ogni capitolato, espone una breve descrizione del prodotto che l'azienda chiede di sviluppare, seguito da una serie di pro e contro emersi in base a criteri soggettivi dei membri del gruppo.
            \\La valutazione è fatta sulla base di:
            \begin{itemize}
                  \item \textbf{Presentazione del capitolato}: una presentazione più curata e dettagliata riscontra maggiore successo in fase di valutazione;
                  \item \textbf{Interesse del gruppo al tema del progetto}: un capitolato porta con sé un tema. Viene valutato quanto ogni capitolato sia interessante, per apportare un impatto positivo allo svolgimento da parte del gruppo;
                  \item \textbf{Tecnologie da utilizzare}: il contesto tecnologico di comune interesse porta maggiore produttività ed entusiasmo all'interno del gruppo;
                  \item \textbf{Conoscenze pregresse}: un capitolato può risultare più o meno complicato da svolgere a seconda delle conoscenze acquisite nel percorso dai membri del gruppo;
                  \item \textbf{Supporto da parte dell'azienda}: maggiore è il supporto offerto dall'azienda, migliore sarà la valutazione del capitolato.
            \end{itemize}
            Il documento offre una panoramica completa di tutti i capitolati, potendoli valutare minuziosamente prima di esprimere una valutazione.
      \item \textit{\textbf{Norme di progetto}}: documento interno che contiene le norme applicate dai membri del gruppo durante il ciclo di vita del prodotto.
            Il corpo di questo documento è composto da:
            \begin{itemize}
                  \item \textbf{Introduzione}: contiene una breve descrizione dello scopo del documento e del contesto in cui viene applicato;
                  \item \textbf{Processi primari}: definisce e norma i processi primari, nel nostro caso:
                        \begin{itemize}
                              \item Fornitura: definisce e norma la fornitura e il rapporto con il proponente;
                              \item Sviluppo: definisce e norma lo sviluppo, per quanto riguarda l'analisi dei
                                    requisiti, la progettazione e la codifica.
                        \end{itemize}

                  \item \textbf{Processi di supporto}: definisce e norma i processi di supporto, nel nostro caso:
                        \begin{itemize}
                              \item Documentazione: descrive le regole e la struttura dei documenti, il ciclo di
                                    vita e le convenzioni da utilizzare durante la stesura. Vengono descritti i
                                    documenti presenti nel repository;
                              \item Gestione della configurazione: descrive il versionamento dei documenti, la
                                    struttura del repository; norma il tracciamento e controllo delle modifiche ai
                                    documenti;
                              \item Accertamento della qualità: descrive come vengono garantite le caratteristiche
                                    di qualità dei processi e dei prodotti;
                              \item Verifica: descrive come vengono verificati i documenti e i prodotti software;
                              \item Validazione: definisce e norma la validazione.
                        \end{itemize}

                  \item \textbf{Processi organizzativi}: definisce e norma i processi riguardanti l'organizzazione del gruppo, in termini di:
                        \begin{itemize}
                              \item Pianificazione: definisce il metodo di lavoro, i ruoli che verranno assegnati e
                                    le responsabilità che da essi ne derivano;
                              \item Modalità di comunicazione: definisce le modalità attraverso le quali il gruppo
                                    comunicherà internamente ed esternamente (qualsiasi comunicazione che comprenda
                                    un soggetto esterno al gruppo di lavoro);
                              \item Modalità di riunione: definisce e norma le modalità con le quali si svolgeranno
                                    le riunioni, interne ed esterne;
                              \item Gestione di infrastrutture: descrive le infrastrutture utilizzate per lo
                                    sviluppo del progetto, affinchè venga garantita affidabilità e sicurezza;
                              \item Gestione dei dubbi o conflitti: spiega come gestire dubbi o conflitti interni
                                    al gruppo.
                        \end{itemize}

                  \item \textbf{Standard di qualità ISO/IEC 9126}: descrive lo standard adottato per garantire la qualità del software prodotto;
                  \item \textbf{Standard di qualità ISO/IEC 12207:1995}: descrive lo standard adottato per garantire la qualità dei processi relativi al ciclo di vita del software;
                  \item \textbf{Metriche}: descrive le metriche adottate per la valutazione quantitativa dei processi e dei prodotti, spiegando
                        il significato di ogni metrica e il modo in cui le valutazioni vengono calcolate.

            \end{itemize}
\end{itemize}
\subsubsection{Documentazione esterna}
La documentazione esterna è composta da tutti i documenti che interessano anche
il proponente e/o il committente.
\begin{itemize}
      \item \textit{\textbf{Verbali esterni}}: verbali frutto di incontri fra i membri del gruppo e soggetti esterni ad esso.\\
            Seguono la struttura dei verbali interni, il nome del file deve avere la forma "VE\_yyyy-mm-dd". Cambia il processo di approvazione\textsubscript{g} del documento:
            il verbale viene approvato non solo dal responsabile, ma anche dal soggetto esterno con il quale si è svolto l'incontro,
            per rendere il documento più professionale e garantire coerenza tra il gruppo e il soggetto esterno.
      \item \textit{\textbf{Lettera di candidatura}}: serve per esprimere la volontà di candidarsi allo svolgimento del capitolato scelto, dopo aver discusso e redatto lo \textit{Studio di fattibilità},
            è composta da:
            \begin{itemize}
                  \item Pagina iniziale: viene dichiarato il capitolato per il quale il gruppo ha
                        deciso di candidarsi, elencando i capitolati che più hanno riscontrato
                        interesse da parte del gruppo, inoltre espone una descrizione di cosa
                        aspettarsi dal contenuto del documento;
                  \item Resoconto degli incontri: una breve descrizione degli incontri svolti con le
                        aziende riguardo i capitolati elencati nella pagina iniziale;
                  \item Motivazione della scelta: espone il motivo per il quale il gruppo ha scelto il
                        capitolato a cui candidarsi.
            \end{itemize}
      \item \textit{\textbf{Diario di bordo}}: documento informale ad uso esterno che permette di interagire settimanalmente con il committente per riportare aggiornamenti sullo stato di
            avanzamento del progetto, descrivendo tutto ciò che è stato fatto rispetto al \textit{Diario di bordo} precedente,
            e quello che il gruppo si impegna a fare nel periodo successivo, vengono inoltre riportati dubbi e domande da porre al committente.
            Vengono elaborati tramite la piattaforma online Canva, che permette di creare presentazioni collaborative, accessibili online.\\
            La struttura del \textit{Diario di bordo} è composta da quattro slide contenenti:
            \begin{itemize}
                  \item \textbf{Slide 1}: slide di presentazione che contiene:
                        \begin{itemize}
                              \item Logo;
                              \item Nome del gruppo;
                              \item Indirizzo email del guppo;
                              \item Titolo del documento: il titolo del \textit{Diario di bordo} segue una sintassi
                                    prefissata ovvero "Diario di bordo \#N", dove N è un numero che incrementa ad
                                    ogni presentazione.
                        \end{itemize}
                  \item \textbf{Slide 2}: contiene ciò che è stato svolto nel periodo trascorso;
                  \item \textbf{Slide 3}: contiene ciò che il gruppo si impegna a portare a termine nel periodo successivo;
                  \item \textbf{Slide 4}: contiene dubbi da chiarire e difficoltà incontrate dal gruppo.
            \end{itemize}
      \item \textit{\textbf{Analisi dei requisiti}}: contiene la descrizione dettagliata dei casi d'uso, dei
            requisiti e definisce in modo chiaro e completo le funzionalità che il prodotto deve offrire.
            \\Il corpo del documento è composto da queste sezioni:
            \begin{itemize}
                  \item \textbf{Introduzione}: riporta lo scopo del documento, lo scopo del prodotto, i riferimenti normativi e informativi;
                  \item \textbf{Casi d'uso}: descrive tutti i possibili scenari di utilizzo da parte dell'utente del prodotto,
                        come specificato nella sottosezione \nameref{inf:UC};
                  \item \textbf{Requisiti}: contiene tutte le aspettative e vincoli definiti dal proponente o ricavati da
                        riunioni interne, come specificato nella sottosezione \nameref{inf:reqs}.
            \end{itemize}
      \item \textit{\textbf{Piano di progetto}}: ha lo scopo di supportare la gestione delle risorse per quanto riguarda l'avanzamento del progetto, per riuscire a portarlo a termine entro la data decisa.\\
            Il \textit{Piano di progetto} ha inoltre la funzione di descrivere il modello di sviluppo adottato, e di monitorarlo tramite la suddivisione in periodi, per analizzare il lavoro svolto e poter apportare miglioramenti con il passare del tempo.
            \\Il corpo del documento è composto da queste sezioni:
            \begin{itemize}
                  \item \textbf{Introduzione}: riporta lo scopo del documento, lo scopo del prodotto, i riferimenti normativi e informativi;
                  \item \textbf{Analisi dei rischi}: riassume i rischi a cui il gruppo si espone a seguito dell'aggiudicazione del capitolato, suddivisi in:
                        \begin{itemize}
                              \item Rischi riguardanti i requisiti;
                              \item Rischi tecnologici;
                              \item Rischi organizzativi;
                              \item Rischi personali.
                        \end{itemize}
                  \item \textbf{Modello di sviluppo}: descrive il modello di sviluppo adottato dal gruppo, che si basa su un approccio agile. Viene fornita una descrizione completa del modello, illustrandone i principi fondamentali, le metodologie applicate e le motivazioni che hanno portato alla scelta di questo approccio;
                  \item \textbf{Pianificazione}: descrive lo svolgimento delle attività suddiviso per periodi, fornendo un riassunto degli obiettivi previsti per ciascuno sprint. Verranno indicati i risultati che il gruppo intende raggiungere e sarà incluso un diagramma di Gantt per visualizzare la pianificazione delle tempistiche e il progresso delle attività;
                  \item \textbf{Preventivo}: viene pianificata in dettaglio la suddivisione dei ruoli con le corrispondenti ore di lavoro, per fornire un preventivo rispetto al periodo a cui ci si sta accingendo;
                  \item \textbf{Consuntivo}: presenta i dati raccolti al termine di ciascun periodo, confrontandoli con le previsioni indicate nella sezione di preventivo. Verranno riportate le ore effettive di lavoro svolte e il relativo costo,
                        accompagnati da un resoconto dettagliato che confronta le stime iniziali con i valori effettivi. Questo confronto permette di valutare eventuali scostamenti e di analizzarne le cause;
                  \item \textbf{Attualizzazione dei rischi}: vengono riportati i rischi che si sono verificati durante lo svolgimento del progetto e le relative misure di mitigazione attuate.
            \end{itemize}

      \item \textit{\textbf{Piano di qualifica}}: descrive i principi guida e le attività messe in atto dal team per assicurare che i processi
            adottati e prodotti sviluppati durante lo svolgimento del progetto siano di alta qualità, in termini di efficienza ed efficacia.\\
            In questo documento verranno presentati i risultati delle analisi quantitative condotte per valutare le performance del team, evidenziando eventuali criticità e azioni correttive intraprese.
            \\Il corpo del documento è composto da queste sezioni:
            \begin{itemize}
                  \item \textbf{Introduzione}: riporta lo scopo del documento, lo scopo del prodotto, i riferimenti normativi e informativi;
                  \item \textbf{Qualità di processo}: viene indicato lo standard adottato per la valutazione dei processi e quello per garantire un miglioramento continuo.\\
                        Ad ogni tipologia di processo (primari, di supporto e organizzativi) viene associata una sottosezione che contiene una tabella con
                        i processi appartenenti a tale tipologia. Per ogni processo, sono indicati:
                        \begin{itemize}
                              \item Nome;
                              \item Breve descrizione;
                              \item Metriche utilizzate per la valutazione.
                        \end{itemize}

                        Segue la sottosezione Metriche, che contiene una tabella con le metriche
                        adottate per la valutazione dei processi, riportandone:
                        \begin{itemize}
                              \item Codice identificativo;
                              \item Nome;
                              \item Valore minimo accettabile;
                              \item Valore ottimo.
                        \end{itemize}
                  \item \textbf{Qualità di prodotto}: viene indicato lo standard di riferimento adottato dal team per garantire la qualità del software e della documentazione prodotta.
                        Ad ogni tipologia di prodotto (documenti e software) viene associata una sottosezione che contiene una tabella con
                        gli obiettivi di qualità relativi a tale tipologia. Per ogni obiettivo, sono indicati:
                        \begin{itemize}
                              \item Nome;
                              \item Breve descrizione;
                              \item Metriche utilizzate per la valutazione.
                        \end{itemize}
                        Segue la sottosezione Metriche, che contiene una tabella con le metriche
                        adottate per la valutazione dei processo, riportandone:
                        \begin{itemize}
                              \item Codice identificativo;
                              \item Nome;
                              \item Valore minimo accettabile;
                              \item Valore ottimo.
                        \end{itemize}
                  \item \textbf{Specifica dei test}: descrive in modo approfondito i test e i risultati ottenuti.
                        Ad ogni tipologia di test (unità, integrazione, sistema, accettazione) viene associata una sottosezione che contiene una tabella con
                        i test relativi a tale tipologia. Per ogni test, sono indicati:
                        \begin{itemize}
                              \item Codice identificativo;
                              \item Breve descrizione;
                              \item Stato.
                        \end{itemize}
                        Segue la sottosezione Tracciamento dei requisiti, contenente una tabella che associa
                        ogni test di sistema a un requisito software.
                  \item \textbf{Resoconto delle attività di verifica}: illustra i dati raccolti durante la valutazione dei processi e dei prodotti.
                        Descrive in forma tabellare, per ogni metrica di interesse, il valore registrato al termine dell'ultimo sprint
                        completato e l'esito della verifica. Per ogni metrica di processo viene usato un grafico per descriverne l'andamento
                        durante il corso del progetto.
            \end{itemize}
      \item \textit{\textbf{Glossario}}: documento che contiene il significato dei termini chiave utilizzati nel progetto (indicati con il pedice "g"), utile per garantire una comprensione comune fornendo spiegazioni concise e precise.
            \\La struttura del \textit{Glossario} è composta da sezioni caratterizzate dalle lettere dell'alfabeto in ordine alfabetico, contenenti le parole che hanno come iniziale quella lettera.
\end{itemize}

\subsubsection{Metriche}
Per perseguire la qualità nel processo di documentazione, si è deciso di
adottare le seguenti metriche:
\begin{itemize}
      \item \nameref{M:DOCC};
      \item \nameref{M:GI};
      \item \nameref{M:GE}.
\end{itemize}

\subsection{Gestione della configurazione}
\subsubsection{Scopo}
La gestione della configurazione è un processo che mira a gestire e controllare
i cambiamenti apportati a un prodotto software o a un sistema durante il suo
ciclo di vita. La gestione della configurazione per la documentazione descrive
come vengono identificate, controllate, tracciate e gestite le versioni di un
documento.
\subsubsection{Versionamento}
Ogni versione del documento è identificata da un codice di versione nel formato
\textbf{Z.Y.X} dove:
\begin{itemize}
      \item \textbf{Z}: il documento è stato approvato dal responsabile;
      \item \textbf{Y}: la sezione aggiunta o modificata di un documento è stata verificata;
      \item \textbf{X}: è stata corretta velocemente qualche incoerenza o errore minore.
\end{itemize}
\subsubsection{Repository}
Il repository si può trovare all'indirizzo
\textbf{\url{https://6bitbusters.github.io}} ed è pubblico. I collaboratori
sono i componenti del gruppo \textit{6BitBusters} che utilizzano il proprio
account GitHub personale per collaborare al progetto. La struttura del
repository è formata in questo modo:
\begin{itemize}
      \item \textbf{ .github}: cartella che contiene i file sorgenti delle action e i template per le issue;
      \item \textbf{3Dataviz}: cartella che contiene i file sorgente del prodotto software;
      \item \textbf{Docs}: cartella che contiene la documentazione. Si divide ulteriormente in:
            \begin{itemize}
                  \item \textbf{Candidatura}: cartella che contiene i documenti da presentare per la candidatura;
                  \item \textbf{Generali}: cartella che contiene tutta la documentazione esterna e interna tranne i verbali;
                  \item \textbf{Verbali esterni}: cartella contenente i verbali esterni, che riportano gli incontri con soggetti esterni;
                  \item \textbf{Verbali interni}: cartella contenente i verbali interni, relativi agli incontri tra membri del gruppo.
            \end{itemize}
      \item \textbf{website}: cartella che contiene i file sorgente del sito web che successivamente verrà pubblicato utilizzando GitHub Pages.
\end{itemize}
Il repository è dotato di sistema di auto-build per la documentazione grazie a GitHub Actions (vedi \nameref{inf:automaz}).
Per maggiori informazioni riguardanti i branch seguire le regole descritte nella sezione \nameref{inf:branch}.

\subsubsection{Branching}\label{inf:branch}
I branch si dividono in:
\begin{itemize}
      \item \textbf{Branch protetti}: sono modificabili solo tramite pull request
            \begin{itemize}
                  \item \textbf{main}: branch principale che contiene la documentazione e il codice approvato del responsabile,
                        nonché la parte che viene mostrata sulla pagina web;
                  \item \textbf{docs/[NOME-DOCUMENTO]}: un branch per ogni documento, che contiene solamente le versioni verificate del documento.
                        Se il nome del documento include dei trattini ("-"), nel nome del branch devono essere sostituiti con underscore ("\_"),
                        ad esempio docs/VI\_2024\_11\_20.
            \end{itemize}
      \item \textbf{Branch derivati}: sono branch utilizzati per aggiungere modifiche e aggiornare un documento o una parte di codice che poi dovrà essere verificata
            attraverso una pull request verso il branch da cui esso è stato derivato;\\
            I nomi di questi branch, per quanto riguarda la documentazione, si suddividono in due casi:
            \begin{itemize}
                  \item \textbf{Per \textit{Glossario}}:
                        \begin{center}
                              \textbf{docs/glossario-[COGNOME-ASSEGNATARIO]}
                        \end{center}
                  \item \textbf{Per tutto il resto della documentazione}:
                        \begin{center}
                              \textbf{docs/[NOME-DOCUMENTO]-[ID-ISSUE]}
                        \end{center}
            \end{itemize}
            dove:

            \begin{itemize}
                  \item \textbf{NOME-DOCUMENTO}: indica il nome del documento sul quale si sta lavorando;
                  \item \textbf{ID-ISSUE}: indica il numero identificativo associato alla issue relativa alla modifica del documento;
                  \item \textbf{COGNOME-ASSEGNATARIO}: indica il cognome del membro che modifica il documento.
            \end{itemize}
            Se un componente richiede una pull request, ma il branch di destinazione è più aggiornato di quello di partenza,
            deve fare un merge dal branch di destinazione a quello di partenza. In questo modo vengono applicate le modifiche delle versioni
            già approvate/verificate nel branch di partenza. Inoltre, è necessario che il registro delle modifiche e la pagina iniziale
            vengano adeguatamente modificati per segnalare l'incremento della versione.
      \item \textbf{Branch di hotfix}: sono branch dedicati all'hotfix e quindi a correzioni minime, sia per la documentazione che per il codice. Valgono le stesse regole dei
            branch protetti, quindi le modifiche devono essere sempre verificate e in caso approvate.
            Il nome per questi branch deve essere:
            \begin{center}
                  \textbf{hotfix/[NOME-DOCUMENTO]}
            \end{center}
\end{itemize}

In ogni branch derivato il lavoro può essere svolto da un solo componente del
gruppo, specificatamente colui che ha preso in carico la issue alla cui
risoluzione quel branch è dedicato.

Una volta che la issue viene risolta il componente deve richiedere una pull
request verso il branch da cui esso deriva. Notare che se il branch di
destinazione è:
\begin{itemize}
      \item \textbf{main} $\rightarrow$ il documento deve essere approvato;
      \item \textbf{docs/[NOME-DOCUMENTO]} $\rightarrow$ il documento deve essere verificato e successivamente viene eliminato il branch derivato.
\end{itemize}

Di seguito si riporta un semplice ma esaustivo esempio di workflow tramite
immagine:
\begin{center}
      \includegraphics[scale = 0.33]{template/images/workflow.png}
\end{center}

\subsubsection{Commit}\label{inf:comm}
È preferibile che ogni commit abbia una singola responsabilità per cambiamento.
I commit non possono essere effettuati direttamente sui branch protetti ma per contribuire con delle aggiunte o
modifiche sarà necessario aprire una pull request, motivo per il quale abbiamo introdotto dei branch derivati.
I messaggi di commit dovranno seguire la seguenti strutture sintattiche:
\begin{center}
      \textbf{add: [COSA]\\
      change: [COSA]\\
      restructure: [COSA]}
\end{center}
dove:
\begin{itemize}
      \item \textbf{add}: viene utilizzato quando si va ad aggiungere una nuova sezione/sottosezione a un documento;
      \item \textbf{change}: viene utilizzato quando si va a modificare una o più sezioni/sottosezioni di un documento;
      \item \textbf{restructure}: indica una ristrutturazione del branch main per quanto riguarda l'organizzazione delle cartelle e/o la modifica di action o template di issue;
      \item \textbf{COSA}: breve descrizione di cosa di è aggiunto e/o fatto.
\end{itemize}

Notare però che dopo l'approvazione di una pull request tutti i commit relativi
verranno raggruppati in un unico commit, il cui titolo deve rispettare la
struttura sintattica descritta in seguito.
\begin{center}
      \textbf{Update: [NOME-DOCUMENTO]-[VERSIONE]}
\end{center}
dove:

\begin{itemize}
      \item \textbf{NOME-DOCUMENTO}: indica il nome del documento nel quale sono state verificate le modifiche;
      \item \textbf{VERSIONE}: indica il numero di versione aggiornata.
\end{itemize}
Per quanto riguarda il commento facoltativo, si lascia quello di default proposto da GitHub,
ovvero un elenco puntato di tutti i commit che verranno raggruppati.\\

Per aggiornare direttamente il main si è deciso di disabilitare temporaneamente
il protection, eseguire un commit con delle modifiche, con la struttura
sintattica elencata sopra, e infine attivare nuovamente il protection. Questa
soluzione è stata adottata perché il team ritiene sia più veloce e semplice
apportare modifiche a template e action in questo modo rispetto alla creazione
di un branch e a una successiva pull request, inoltre il main, una volta
ultimate le automazioni, non verrà più modificato se non dai merge scatenati
dalle pull request.

\subsubsection{Pull request}\label{inf:pr}
Per effettuare un merge su un branch protetto si deve aprire, da GitHub, una
pull request. Questa permette di verificare il lavoro svolto prima di
integrarlo con un branch protetto ed eseguire un veloce test di compilazione
della sezione aggiunta. Alla creazione di una pull request bisogna associare:
\begin{itemize}
      \item \textbf{Title}: [NOME-DOCUMENTO]-[ID-ISSUE];
      \item \textbf{Verificatori in carica}: coloro che hanno il compito di trovare eventuali errori o mancanze e fornire un feedback
            riguardante il contenuto direttamente su GitHub attraverso un commento, sulla stessa pull request.
            Non sarà possibile effettuare il merge finché tutti i commenti di revisione non saranno stati risolti
            con, al termine, l'approvazione di almeno uno dei verificatori e il test di build non dia esito positivo;
      \item \textbf{Descrizione}: contiene una lista riassuntiva di ciò che è stato fatto includendo le issue completate, e quindi da chiudere,
            con la sintassi
            \begin{center}
                  \textbf{close \#[ID-ISSUE]}
            \end{center}
            in modo tale che vengano tutte chiuse in automatico quando la pull request verrà accettata;
      \item \textbf{Gli assegnatari}: coloro che hanno anche il compito di apportare le modifiche necessarie al documento;
      \item \textbf{Label}: riassume di che natura è la pull request.
\end{itemize}
dove:

\begin{itemize}
      \item \textbf{NOME-DOCUMENTO}: indica il nome del documento sul quale si sta lavorando e richiedendo la verifica/approvazione;
      \item \textbf{ID-ISSUE}: indica il numero identificativo associato alla issue relativa alla modifica del documento.
\end{itemize}

Per una descrizione più dettagliata delle issue si faccia riferimento a
\nameref{inf:its}.

Per i commit relativi alle pull request seguire le regole descritte nella
sezione \nameref{inf:pr}.

\subsubsection{Automazione}\label{inf:automaz}
\subsubsubsection{Scopo}
Lo scopo dell'automazione è garantire una maggiore efficienza e ridurre il
rischio di errori umani. Tuttavia, il costo necessario per rendere automatiche
alcune attività potrebbe essere superiore al beneficio ottenuto. È quindi
fondamentale scegliere attentamente quali processi automatizzare.
\subsubsubsection{Automazioni realizzate}
L'automazione dei processi di verifica e di compilazione della documentazione
avviene tramite GitHub Actions, che permette di eseguire script personalizzati
in risposta a eventi specifici che si verificano nel repository. In
particolare, sono state scritte tre diverse action:
\begin{itemize}
      \item \texttt{build.yml}: si scatena a seguito di un push sul main; ha il compito di compilare
            tutta la documentazione creata fino a quel momento e di creare la relativa pagina web tramite Pages;
      \item \texttt{pr\_build\_check.yml}: si scatena quando viene effettuata una pull request verso il main o verso
            un branch di documentazione, ovvero del tipo docs/[NOME-DOCUMENTO]. Serve per verificare che
            i sorgenti LaTeX compilino correttamente e che sia stato modificato uno e un solo documento;
      \item \texttt{pr\_gulpease\_check.yml}: si scatena quando viene effettuata una pull request verso il main o verso
            un branch di documentazione, ovvero del tipo docs/[NOME-DOCUMENTO]. Essa verifica che
            l'indice Gulpease del documento, espresso come media degli indici delle sezioni, sia maggiore o uguale a 50.
\end{itemize}
Per garantire la qualità dei documenti prodotti, è necessario che le action \texttt{pr\_build\_check.yml}
e \\\texttt{pr\_gulpease\_check.yml} diano esito positivo. A tale scopo, è stato configurato il repository in modo
che non sia possibile effettuare il merge di una pull request se non sono soddisfatti entrambi i controlli.
Inoltre, per accelerare il processo di verifica, è fortemente raccomandato eseguire localmente gli script di test,
in modo da individuare e correggere eventuali errori prima di effettuare la pull request.
\subsubsubsection{Aggiungere un'action}
Per aggiungere un'action, è necessario seguire i seguenti passaggi:
\begin{itemize}
      \item Creare un file in linguaggio YAML nella cartella \texttt{.github/workflows} del
            repository, in cui si specificano i dettagli dell'action. In particolare, è
            necessario definirne:
            \begin{itemize}
                  \item Nome;
                  \item Evento che la scatena;
                  \item Job da eseguire;
                  \item Azioni da compiere all'interno di ciascun job.
            \end{itemize}
      \item Se necessario al funzionamento dell'automazione, produrre uno o più script e
            inserirli nella cartella \texttt{.github}. Il linguaggio utilizzato per
            scrivere tali script è python, scelto per la sua comprensibilità e facilità di
            utilizzo;
      \item Disabilitare temporaneamente il protection del main;
      \item Effettuare un commit con le modifiche apportate (vedi \nameref{inf:comm});
      \item Riabilitare il protection del main.
\end{itemize}

\subsection{Accertamento della qualità}
\subsubsection{Scopo}
Lo scopo del processo di accertamento della qualità è garantire la qualità dei
processi adottati e del software prodotto, al fine di soddisfare le aspettative
degli stakeholder e i requisiti del progetto. Ciò avviene tramite le attività
di pianificazione della qualità, il controllo della qualita e il miglioramento
continuo.
\subsubsection{Descrizione}
Il piano della qualità comprende gli obiettivi di qualità che il team si
impegna a raggiungere e le metriche utilizzate per misurare il raggiungimento
di tali obiettivi. \\ Il controllo della qualità consite nella misurazione e
analisi degli indicatori di interesse, valutando il grado di raggiungimento
degli obiettivi di qualità. \\ Il documento \textit{Piano di qualifica} tratta
nel dettaglio le attività di pianificazione e controllo della qualità. \\
\subsubsection{Ciclo di Deming}
Per quanto riguarda l'attività di miglioramento continuo dei processi, il
gruppo adotta il ciclo di Deming. Si tratta di un metodo di gestione iterativo
che prevede quattro fasi:
\begin{itemize}
      \item \textbf{Plan}: pianificazione degli obiettivi e dei processi necessari per fornire
            risultati in accordo con i risultati attesi, attraverso la creazione di attese di
            produzione, di completezza e accuratezza delle specifiche scelte;
      \item \textbf{Do}: attuazione del piano, tramite l'esecuzione dei processi,
            lo sviluppo dei prodotti e la misurazione dei risultati ottenuti;
      \item \textbf{Check}: analisi e studio dei dati raccolti nella fase Do.
            I grafici possono rendere più facile il confronto tra risultati ottenuti e attesi;
      \item \textbf{Act}: consolidamento delle soluzioni che hanno dato buoni risultati e
            adozione di strategie correttive per migliorare ciò che non ha soddisfatto le aspettative,
            dopo averne compreso a fondo le cause. In questo modo, ogni iterazione del ciclo di Deming
            contribuisce ad aggiungere valore al processo.
\end{itemize}
\subsubsection{Metriche}
Per perseguire la qualità nel processo di accertamento di qualità, si è deciso
di adottare le seguenti metriche:
\begin{itemize}
      \item \nameref{M:MM}.
\end{itemize}
\subsection{Verifica}
\subsubsection{Scopo}
Lo scopo del processo di verifica\textsubscript{g} è quello di accertare che
non siano stati commessi errori nello svolgimento delle attività prefissate.
Questo processo viene applicato costantemente sia durante la stesura della
documentazione, che durante lo sviluppo del software.
\subsubsection{Descrizione}
La verifica verrà fatta in maniera efficace adottando questi metodi:
\begin{itemize}
      \item \textbf{Analisi statica}: non richiede esecuzione dell'oggetto di verifica, quindi applicabile a documentazione e codice, usata per accertare conformità e regole nonché assenza di difetti, adotteremo i due seguenti metodi di lettura:
            \begin{itemize}
                  \item \textbf{Walkthrough}: il verificatore esamina documenti o codice in cerca di difetti, ma senza svolgere una
                        ricerca specifica per un certo tipo di errore. L'obiettivo è identificare errori, migliorare la qualità e favorire la comprensione;
                  \item \textbf{Inspection}: il verificatore controlla documenti o codice per trovare errori
                        eseguendo un'analisi mirata dell'oggetto di verifica. È necessario organizzare gli elementi da verificare in liste di controllo,
                        per poi passare alla verifica vera e propria. Questo approccio è più efficiente rispetto al walkthrough, 
                        anche perché in molti casi può essere automatizzato.
            \end{itemize}
      \item \textbf{Analisi dinamica}: richiede l'esecuzione dell'oggetto di verifica, quindi è applicabile solo al codice, usata per accertare che il codice sia corretto, non introduca errori nel sistema e soddisfi i requisiti preposti.
\end{itemize}
\subsubsection{Verifica della documentazione} Al momento della necessità di modificare o aggiungere qualcosa a una sezione di
un documento, si procederà in questo modo:
\begin{enumerate}
      \item Verrà creata una issue che specifica l'attività di modifica o aggiunta da
            svolgere;
      \item Verrà creato un sotto-branch\textsubscript{g} chiamato
            "docs/[NOME-DOCUMENTO]-N" dove con N si intende il numero della issue di
            riferimento;
      \item Rimanendo in questo sotto-branch\textsubscript{g}, verrà aggiornato il
            documento;
      \item Una volta terminato, per iniziare il processo di verifica\textsubscript{g}
            verrà aperta una pull request\textsubscript{g};
      \item Uno tra i verificatori in carica durante quello sprint si assegnerà la verifica
            del documento e modificherà il registro delle modifiche compilando il campo
            "Verificatore";
      \item Se la verifica avrà esito positivo, ci sarà il merge tra il
            sotto-branch\textsubscript{g} e il branch\textsubscript{g} principale del
            documento, con conseguente chiusura della issue e del sotto-branch;
      \item In caso di esito negativo, verranno segnalati gli errori da parte del
            verificatore tramite commenti, e il redattore si occuperà di risolverli
            continuando a lavorare nel sotto-branch\textsubscript{g}, fino ad avere il
            documento verificato.
\end{enumerate}

\subsubsection{Verifica del codice}
Il processo di verifica del codice si attua principalmente attraverso lo
sviluppo e l'esecuzione di test, che permettono di eseguire un'analisi dinamica
del software, ossia un'analisi che si concentra sul comportamento del programma
durante l'esecuzione. I test vengono progettati per garantire che il codice
funzioni correttamente in vari scenari e che rispetti i requisiti prefissati. I
principali tipi di test utilizzati in questo processo sono:
\begin{itemize}
      \item \textbf{Test di unità}: verificano il corretto funzionamento delle singole unità o funzioni del codice, testando piccoli blocchi di logica isolati.
            Questi test sono fondamentali per individuare errori precoci e garantire che ogni componente funzioni come previsto.
            Vengono definiti durante la progettazione di dettaglio;
      \item \textbf{Test di integrazione}: verificano che i vari moduli o componenti del sistema interagiscano correttamente tra loro.
            Questi test sono particolarmente utili per identificare problemi che si verificano quando le diverse parti del sistema vengono messe insieme.
            Vengono definiti nella fase di progettazione logica;
      \item \textbf{Test di regressione}: vengono eseguiti per verificare che le modifiche al codice non abbiano introducano nuovi errori o causino anomalie in funzionalità già presenti.
            Questi test sono essenziali per mantenere l'integrità del software durante lo sviluppo continuo;
      \item \textbf{Test di sistema}: testano il comportamento complessivo dell'applicazione in un ambiente che simula il più possibile l'ambiente di produzione.
            Questi test verificano che l'intero sistema funzioni come previsto sotto condizioni realistiche e che tutte le componenti siano correttamente integrate.
            Vengono definiti nella fase di analisi dei requisiti software;
\end{itemize}
I test vengono identificati attraverso un codice con questa struttura:
\textbf{
      \[
            T[\text{Tipo}][ \text{N}]
      \]
}
\\Dove N è il codice del test ed è un valore univoco e il tipo può essere:
\begin{itemize}
      \item \textbf{U}: test di unità;
      \item \textbf{I}: test di integrazione;
      \item \textbf{R}: test di regressione;
      \item \textbf{S}: test di sistema.
\end{itemize}
Ogni test, inoltre, possiede uno stato, che può essere:
\begin{itemize}
      \item \textbf{NI}: non implementato;
      \item \textbf{S}: superato;
      \item \textbf{NS}: non superato.
\end{itemize}
\paragraph{Metriche}
Per perseguire la qualità nel processo di verifica, si è deciso di adottare le
seguenti metriche:
\begin{itemize}
      \item \nameref{M:COC};
      \item \nameref{M:SC};
      \item \nameref{M:BC};
      \item \nameref{M:PTC};
      \item \nameref{M:LT};
      \item \nameref{M:ART};
      \item \nameref{M:CYC};
      \item \nameref{M:PPM};
      \item \nameref{M:LPM};
      \item \nameref{M:LPF};
      \item \nameref{M:CD}.
\end{itemize}
\subsection{Validazione}
Nel processo di validazione il fornitore dimostra che tutti i requisiti utente
sono stati soddisfatti. A tal fine assumono fondamentale importanza
i test di accettazione. Essi verificano che il prodotto finale sia conforme ai requisiti e
alle aspettative dell'utente finale, e quindi sia pronto per essere rilasciato.
I test di accettazione vengono definiti con il proponente nella fase di analisi dei requisiti.
